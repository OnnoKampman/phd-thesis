% uncomment for ELECTRONIC simple version
% \chapter*{Abstract}

% uncomment for both ELECTRONIC and HARDBOUND final version
\cleardoublepage
\begin{center}
    \vspace*{2em}
    { \Large {\bfseries Robust time-varying functional connectivity estimation and its relevance for depression} \par}
    {{\large \vspace*{1em} Onno Pepijn Kampman} \par}
    \vspace{2em}
\end{center}

This thesis investigates how to robustly estimate \gls{tvfc}, a construct in neuroimaging research that looks at changes in functional coupling (correlation between time series) between brain regions during a \gls{fmri} scan, and how it can be used as a lens through which to study depression as a functional disorder.

Unfortunately, the field of \gls{tvfc} is still riddled with uncertainty, especially regarding its estimation.
This is mainly due to the absence of a ground truth.
Without resolving this first, the value of any study, including this depression study, is significantly undermined and conclusions made therein less trustworthy.
Therefore, I propose a novel and principled method for estimating \gls{tvfc}, based on the \gls{wp}, a covariance matrix stochastic process that has recently become computationally tractable, and introduce a comprehensive benchmarking framework based on machine learning principles to make sure it performs better than existing methods in the field.
These benchmarks include simulations, subject phenotype prediction, test-retest studies, brain state analyses, external task prediction, and a range of qualitative method comparisons.
Furthermore, I introduce a benchmark based on cross-validation, that can be run on any data set.
The \gls{wp} model is found to outperform other common estimation methods, such as \gls{sw} approaches and \gls{dcc}.

Returning to the depression study, several differences are found between depressed and healthy control cohorts.
The study is run on thousands of participants from the UK Biobank, yielding unprecedented statistical power and robustness.
I investigate connectivity between individual brain regions as well as \glspl{fn}.
Depressed participants show decreased global connectivity, and increased connectivity instability (as measured by the temporal characteristics of estimated \gls{tvfc}).
By defining multiple depression phenotypes, I find that brain dynamics are affected especially when patients have been professionally diagnosed or indicated to be depressed during their \gls{fmri} scan, but were less or not at all affected based on self-reported past instances and genetic predisposition.
I show that choosing a different \gls{tvfc} estimation method would have changed our scientific conclusions.
This sensitivity to seemingly arbitrary researcher choices highlights the need for robust method development and the importance of community-approved benchmarking.

I wrap up this thesis with a discussion of results and how this style of work fits into the bigger picture of neuroscientific research, reflect on what has been learned about depression, and posit promising directions for future work.
