% Returns the width of the current document in pts.
% This can be useful when generating figures, so aspect ratios are preserved.
% \showthe\textwidth
% Currently 360.0pt, independent from margin.

% The inputenc package is ignored with utf8 based engines.
% \usepackage[utf8]{inputenc}

\usepackage[
    abbreviate=false,
    backend=biber,                % biber is the default, can use bibtex or bibtex8 as well
    backref=true,                 % whether to add citation page(s) to References
    bibstyle=authoryear,          % can set to 'draft' when writing
    bibwarn=true,
    block=space,                  % none, space, par, nbpar (does not allow page breaks), or ragged
    citestyle=authoryear,         % citation style, e.g. Kampman (2022)
    doi=true,                     % whether to print DOIs in References, true for ELECTRONIC version, should be false for HARDBOUND version
    isbn=false,                   % whether to print International Standard Serial Number (ISSN) in References
    mincitenames=1,
    maxcitenames=3,
    minbibnames=2,
    maxbibnames=10,
    sorting=nyt,                  % 'count' for sorting in order of number of times cited
    url=false                     % whether to print URLs in References
]{biblatex}

\addbibresource{references.bib}

% TODO: how can we print a newline before the DOI?
% \DeclareFieldFormat*{doi}{\printunit{\newline}#1}  % this works and prints the correct DOI, but without the proper formatting
% \DeclareFieldFormat*{doi}{\printunit{\newline}}  % uncomment for HARDBOUND version, this works by replacing the DOI with a newline

\renewcommand*{\nameyeardelim}{\addcomma\space}  % add comma between author and year

\renewcommand*{\bibfont}{\small}                 % smaller font size for References

% The hyperref package transforms citations into hyperlinks.
\usepackage[
    citecolor=Gray,  % `Gray' for ELECTRONIC version, `Black' for HARDBOUND version
    colorlinks,
    linkcolor=MidnightBlue,  % `MidnightBlue' for ELECTRONIC version, `Black' for HARDBOUND version
    pdfborder={0 0 0},
    pdfauthor={Onno Pepijn Kampman},
    pdfkeywords={functional connectivity, time-varying functional connectivity, machine learning, Wishart process, depression, PhD thesis, University of Cambridge},
    pdfsubject={This thesis studies robust estimation of time-varying functional connectivity using Wishart processes and its utility for understanding depression.},
    pdftitle={Robust time-varying functional connectivity estimation and its relevance for depression},
    urlcolor=RoyalBlue    % `RoyalBlue' for ELECTRONIC version, `Black' for HARDBOUND version
]{hyperref}

% REFERENCES
%   https://mirrors.concertpass.com/tex-archive/macros/latex/contrib/biblatex/doc/biblatex.pdf


%%%%%%%%%
% BOXES %
%%%%%%%%%

\usepackage{tcolorbox}
\newtcolorbox[auto counter]{mybox}[2][]{
  float,
  fontupper=\footnotesize,
  fontlower=\footnotesize,
  title={Box~\thetcbcounter: #2},
  #1
}

% REFERENCES
%   https://en.wikibooks.org/wiki/LaTeX/Colors
%   e.g. define a color, then add colback=my-blue
%   \definecolor{my-blue}{cmyk}{0.80, 0.13, 0.14, 0.04, 1.00}

%%%%%%%%
% MATH %
%%%%%%%%

\usepackage{amsfonts}       % blackboard math symbols
\usepackage{amsmath}
\usepackage{amsthm}

%%%%%%%%%%%
% GENERAL %
%%%%%%%%%%%

\usepackage[toc,page]{appendix}
\usepackage[nottoc]{tocbibind}  % the nottoc option removes Contents from Contents
%%
\usepackage{booktabs}       % professional-quality tables
\usepackage[font=small,labelfont=it]{caption}  % small is 11pt when normalsize is 12pt
%%
\usepackage{cleveref}       % CAUTION: must be loaded after amsmath
%%
% \usepackage{mathptmx}       % use Times New Roman
\usepackage[T1]{fontenc}    % use 8-bit T1 fonts
%%
% The printing company recommended equal margins on all pages; so no binding offset.
\usepackage[
    left=30mm,
    right=30mm,
    top=35mm,
    bottom=30mm
]{geometry}
\usepackage{graphicx}
% \usepackage{lettrine}       % for dropped capital letters (2 lines high)
\usepackage{mathtools}
%%
\usepackage{microtype}      % microtypography (improves visual appearance)
\usepackage{nicefrac}       % compact symbols for 1/2, etc.
% \usepackage{pdfpages}       % for including PDF pages
\usepackage{setspace}       % define line spacing in paragraph
%%
\usepackage[labelfont=bf,textfont=normalfont]{subcaption}     % allows for subplots
%%
\usepackage{xargs}          % Use more than one optional parameter in a new commands
\usepackage{url}            % simple URL typesetting

%%%%%%%%%%%%%%%%%%%%%
% HEADERS & FOOTERS %
%%%%%%%%%%%%%%%%%%%%%

\usepackage{fancyhdr}             % Includes header on each page
\setlength{\headheight}{14.5pt}   % Required to avoid overfull vbox warnings (13.6pt for 11pt, 14.5 for 12pt)
\fancyhead{}
\fancyfoot{}

% \fancyhead[LE]{\thepage}  % uncomment for final (HARDBOUND) two-sided version
% \fancyhead[RO]{\thepage}  % uncomment for final (HARDBOUND) two-sided version
\fancyhead[R]{\thepage}   % uncomment for one-sided draft and final (ELECTRONIC) two-sided version

% REFERENCES
%   https://www.overleaf.com/learn/latex/How_to_Write_a_Thesis_in_LaTeX_(Part_2)%3A_Page_Layout

%%%%%%%%%
% OTHER %
%%%%%%%%%

\usepackage[
  obeyDraft,           % uncomment for final version (both ELECTRONIC and HARDBOUND); if enabled, the todo notes will only show when running a draft
  colorinlistoftodos,
  prependcaption,
  textsize=tiny        % recommended: footnotesize, scriptsize, or tiny (the smallest)
]{todonotes}

\newcommandx{\unsure}[2][1=]{\todo[linecolor=red,backgroundcolor=red!25,bordercolor=red,#1]{#2}}
\newcommandx{\change}[2][1=]{\todo[linecolor=blue,backgroundcolor=blue!25,bordercolor=blue,#1]{#2}}
\newcommandx{\info}[2][1=]{\todo[linecolor=OliveGreen,backgroundcolor=OliveGreen!25,bordercolor=OliveGreen,#1]{#2}}
\newcommandx{\improvement}[2][1=]{\todo[linecolor=Plum,backgroundcolor=Plum!25,bordercolor=Plum,#1]{#2}}
\newcommandx{\thiswillnotshow}[2][1=]{\todo[disable,#1]{#2}}

% REFERENCES
%   http://tug.ctan.org/macros/latex/contrib/todonotes/todonotes.pdf


\usepackage{lineno}
\onehalfspacing
% \linespread{1.25}  % this is equal to 1.5 linespacing in Microsoft Word
% \doublespacing

\author{Onno Pepijn Kampman}
\date{September 2022}
