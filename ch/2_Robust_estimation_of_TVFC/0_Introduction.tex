\chapter{Robust estimation of TVFC}\label{ch:methods}
%%%%%

\info[inline]{Paragraph: Overview of chapter.}
In this chapter we frame and define the problem of \gls{tvfc} estimation.
We describe key established methods for this task and the development of new ones; the \gls{wp} in particular.
Additionally, we motivate the importance and describe several ways of extracting features or \emph{(bio)markers} from \gls{tvfc}.
These will be used in further analyses in the following chapters.
%
Furthermore, we discuss how to compare and evaluate these methods, to weigh which one we ought to use.
Our aim is to develop \emph{robust} \gls{tvfc} estimation methods.
Above all, we need to convince \emph{ourselves} that any method we opt to use is valid.
In the words of the legendary physicist Richard~Feynman: ``The first principle is that you must not fool yourself, and you are the easiest person to fool''.
%
This chapter closes with a brief discussion on the nature of estimation methods.

\info[inline]{Paragraph: Frame TVFC estimation as covariance structure estimation problem.}
The estimation of \gls{tvfc} (as we have narrowly defined it in the introduction to this thesis) is a particular form of the more general problem of covariance structure estimation.
This problem has been of interest to statisticians and computer scientists for a long time.
It has been studied extensively in the machine learning and econometrics (e.g.~modern portfolio theory) communities.
%
However, each application domain has its own unique characteristics.
Directly copying approaches and methods is not sufficient.
For example, those who study financial markets are mostly interested in using covariance structure modeling to estimate \emph{future} events and returns.
Neuroscientists typically are not interested in such forecasting.

\info[inline]{Paragraph: Introduce concept of covariance.}
To recap, the \textbf{covariance} between two random variables $X$ and $Y$ is defined as the expected value of the product of the deviations of these variables from their individual expected values (i.e.~means):
\begin{equation}
  \begin{aligned}
    \sigma_{xy} & = \mathbb{E}[(X - \mathbb{E}[X])(Y - \mathbb{E}[Y])] \\ & = \mathbb{E}[XY] - \mathbb{E}[X]\mathbb{E}[Y].
  \end{aligned}
  \label{eq:covariance}
\end{equation}
It is a \emph{linear} measure of joint variability of random variables.
A \textbf{covariance matrix} ($D = 2$ dimensional in this case; a.k.a.~variance matrices) collects all such covariances and is defined as
\begin{equation}
  \mathbf{\Sigma} = \begin{bmatrix}
    \sigma_x^2  & \sigma_{xy} \\
    \sigma_{xy} & \sigma_y^2
  \end{bmatrix},
\end{equation}
where~$\sigma_x^2$ and~$\sigma_y^2$ are the respective \textbf{variances} of the random variables, and $\sigma_x$ and $\sigma_y$ are called the respective \textbf{standard deviations}.
Covariance matrices are required to be symmetric (i.e.~$\mathbf{\Sigma} = \mathbf{\Sigma}^T$) and positive semi-definite (having only positive eigenvalues, i.e.~$\mathbf{v}^T \mathbf{\Sigma} \mathbf{v} \geq 0$ for all vectors $\mathbf{v} \in \mathbb{R}^D$).
The converse is true as well; every symmetric, positive semi-definite matrix is a covariance matrix.
The inverse of a covariance matrix $\mathbf{\Sigma}^{-1}$ is called a precision matrix.

\info[inline]{Paragraph: Introduce concept of Pearson correlation.}
As discussed previously, in the study of \gls{fc} it is more common to use Pearson correlation coefficients instead of covariance.
These are \emph{dimensionless} measures of linear dependence.
They can be understood as a normalized version of covariance.
As such, they are more interpretable and easier to work with.
Correlation values are bound to range~$[-1,1]$, where a correlation of~1~indicates perfect correlation (i.e.~it is possible to draw a straight line that perfectly fits all data),~-1~indicates perfect anti-correlation, and~0~indicates uncorrelation (which does \emph{not} imply independence).
The Pearson correlation coefficient can be interpreted as a measure of how well the optimal linear function describing the random variable relationship fits the data.
A (Pearson) \textbf{correlation matrix} is defined as
\begin{equation}
  \label{eq:correlation-matrix}
  \mathbf{\Sigma} = \begin{bmatrix}
    1 & \frac{\sigma_{xy}}{\sigma_x\sigma_y} \\
    \frac{\sigma_{xy}}{\sigma_x\sigma_y} & 1
  \end{bmatrix}.
\end{equation}
If all random variables considered have standard deviation (and variance) equal to~1, the covariance and correlation matrix are identical.\footnote{In fact, this is often the case, as normalizing time series variance is common practice in \gls{fc} studies~\parencite[see e.g.][]{Beckmann2004, Allen2014}.}
Throughout this thesis, \gls{fc} refers to this correlation metric.
All plots will show correlation estimates instead of covariance estimates.
%
We will refer to a single \gls{fc} correlation matrix as a connectivity \emph{state}, and refer to this correlation matrix as a function of time (i.e.~\gls{tvfc}) as covariance or correlation \emph{structure} (used interchangeably).
%
On a cautionary note, using the Pearson correlation as connectivity measure may be too simple.
It assumes that time series are homoscedastic, meaning the variance across a brain scan is homogenous.
Furthermore, this measure is known to be heavily skewed by outliers.
It also does not make any claims of \emph{directionality}.

\info[inline]{Paragraph: Outline of this chapter.}
The rest of this chapter is organized as follows.
Firstly, we discuss established and commonly used \gls{tvfc} estimation methods in the field, and our particular implementation of them.
Then, we discuss how to run the \gls{sw} method in a principled and improved way using cross-validation.
Afterwards, we discuss a proposed new method for estimating \gls{tvfc}: the \gls{wp}.
Then, we discuss how a range of features~\parencite[or \emph{biomarkers} as they are often called in the context of neuroimaging, see e.g.][]{Woo2017, Douw2022} can be extracted from such estimated \gls{tvfc} and the estimation methods discussed.
%
Finally, we discuss a framework of benchmarks by which we can robustly assess and compare method performance.
%
We quickly discuss key methodological issues and controversies before moving on to the empirical studies.
