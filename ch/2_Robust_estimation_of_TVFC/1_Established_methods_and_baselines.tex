\clearpage
\section{Established methods and baselines}\label{sec:established-methods}
%%%%%
\info[inline]{Section: Introduce key and established TVFC estimation methods.}

In this section we briefly describe established, state-of-the-art, and popular \gls{tvfc} estimation methods in the field.
Only methods that are used as baseline methods for comparison in our benchmarks and experiments are discussed in more detail.

%%
\subsection{Static functional connectivity}
%%

\info[inline]{Paragraph: Introduce static functional connectivity estimation.}
Any proposed \gls{tvfc} estimation method has to be compared with a \emph{static} (time-invariant) analysis.
The \gls{sfc} approach makes the implicit assumption that the covariance between random variables of interest (e.g.~brain region time series) does not change across time (i.e.~any arbitrary length of scanning session).
Statistically, it assumes that node time series are \emph{stationary}, and thus that connectivity can be considered a \emph{global} property.
This approach extracts the \emph{general} functional architecture of a particular subject, instead of the moment-to-moment functional interactions.
%
The inclusion of \gls{sfc} in any analysis serves as a sanity check.
The mean of \gls{tvfc} estimates should be similar to the \gls{sfc} estimate.
Furthermore, if a \gls{tvfc} estimation method cannot outperform the \gls{sfc} estimate on some task, this may either indicate that there is no (relevant) dynamic signal in the data set, or that the estimation method is flawed.

\info[inline]{Paragraph: Describe our static functional connectivity estimation.}
A standard covariance \gls{sfc} approach simply computes the covariance between node time series (i.e.~regional activity) across the entire brain scan duration as in \cref{eq:covariance}:
\begin{equation}
  \begin{aligned}
    \sigma_{ij} & = \mathbb{E}[(y_i - \mathbb{E}[y_i])(y_j - \mathbb{E}[y_j])] \\ & = \frac{1}{N} \sum_{n=1}^N (y_{i,n} - \bar y_i)(y_{j,n} - \bar y_j),
  \end{aligned}
\end{equation}
where $\sigma_{ij}$ is the covariance between variables $y_i$ and $y_j$ (two different brain region \gls{bold} time series) for their $N$ observations in the entire time series, for $1 \leq i \leq D$ and $1 \leq j \leq D$.
The respective Pearson correlation coefficient is then given by
\begin{equation}
  \label{eq:sfc-estimation}
  \rho_{ij} = \frac{\sigma_{ij}}{\frac{1}{N} \sqrt{\sum_{n=1}^N (y_{i,n} - \bar y_i)^2 \sum_{n=1}^N (y_{j,n} - \bar y_j)^2}}.
\end{equation}
When $D > 2$, the full estimated correlation matrix can be estimated by looping over all edges in pairwise fashion.
Although \gls{sfc} estimates in neuroimaging are typically computed as described here, there are many other versions available, for example using sparsity-inducing penalties~\parencite{Foti2019}.
%
Since we assume (or enforce) all $\bar y_i$ to be zero, perhaps the simplest connectivity estimate is the covariance matrix \gls{mle}:
\begin{equation}
  \hat{\mathbf{\Sigma}}^{MLE} = \frac{1}{N} \sum_{n=1}^N \mathbf{y}_{n} \mathbf{y}_{n}^T.
\end{equation}
However, in this work we use the unbiased estimator instead:
\begin{equation}
  \hat{\mathbf{\Sigma}} = \frac{1}{N - 1} \sum_{n=1}^N \mathbf{y}_{n} \mathbf{y}_{n}^T.
\end{equation}
For larger values of $N$ this difference becomes negligible.

%%
\subsection{Sliding-windows functional connectivity}\label{subsec:sliding-windows-fc}
%%

\info[inline]{Paragraph: Introduce sliding-windows functional connectivity estimation.}
Albeit criticism, \gls{tvfc} estimation methods based on \gls{sw}~\parencite{Chang2010, Sakoglu2010, Allen2014, Shakil2016, Preti2017} are still the most used throughout the neuroscience literature~\parencite{Lurie2020}.
%
This approach slides (or \emph{rolls}) a time window of a certain size (length)~$w$ and shape (e.g.~square, Gaussian) across the observations (typically with a step size of a single volume), and estimates the covariance or correlation as for the \gls{sfc} case in \cref{eq:sfc-estimation} for each step.
As such it can be considered a (weighted) moving average.
The benefits of using \gls{sw} include that it is well-established, computationally cheap, and simple to implement.

\info[inline]{Paragraph: Discuss variations on sliding-windows functional connectivity.}
Many variations on this method exist and how to best implement \gls{sw} methods has been studied extensively~\parencite[see e.g.][]{Mokhtari2019, Vergara2019}.
One popular way to reduce sensitivity to outliers is \emph{tapered} \gls{sw}.
This approach assigns less weight to observations further away from the center of the window~\parencite{Allen2014, Lindquist2014}.
Effectively this just changes the shape of the window from rectangular/square into e.g.~a Gaussian curve.
%
Rules of thumb have been established regarding window length~$w$ and necessary data preprocessing steps.
%
The lack of consistency across studies in implementation of \gls{sw} is sub-optimal as it makes comparison across studies harder.
Furthermore, stacking heuristics does not scale.
Typically, this situation is when we should start applying machine learning techniques~\parencite[][built this case beautifully]{Zinkevich2015}.

\info[inline]{Paragraph: Discuss problems with sliding-windows functional connectivity.}
The main problem with this method is that without knowing the underlying process (i.e.~covariance structure from brain dynamics), it is hard to pick the right window length.
%
The \gls{sw} approach is also not a \emph{model-based}~\parencite{Foti2019} or \emph{data-driven} approach.\footnote{In neuroscientific context, `data-driven' refers to extracting insights directly from data in a relatively unbiased manner. It is often juxtaposed to `hypothesis-driven' or `theory-driven'.}
Furthermore, the desired behavior at the start and end of the time series is not clear.
See \textcite{Lindquist2014, Leonardi2015, Hindriks2016} for more important nuances and pitfalls of \gls{sw} methods.

\info[inline]{Paragraph: Discuss our particular implementation.}
In all experiments and benchmarks that follow in this thesis we implement a standard \gls{sw} approach to mimic a typical (often non-technical) investigator interested in using the construct of \gls{tvfc} to study the brain.
Researchers are recommended to use a window length between 30 and 60 seconds~\parencite{Shirer2012}.
Both of these window lengths are implemented to test the limit cases.
We implement the rectangular (non-tapered) window, with a step size of a single volume.
We follow the rule of thumb proposed by \textcite{Leonardi2015} and high-pass filter the data to remove frequency components below $\frac{1}{w}$ before running the \gls{sw} algorithm.\footnote{\textcite{Smith2012} and \textcite{Hutchison2013} made similar suggestions.}
Zeros are padded to the start and end of node time series to allow for computing the correlation coefficients around those locations.

%%
\subsection{Multivariate GARCH}
%%

\info[inline]{Paragraph: Introduce MGARCH.}
A commonly used and in many applications unsurpassed algorithm for modeling time-varying covariance structures is the multivariate version of the \gls{garch}~\parencite{Engle1982, Bollerslev1986} process.
This class of methods was first described and used in econometrics.\footnote{In this field they are commonly referred to as multivariate volatility models. In fact, methods for analyzing economic time series with time-varying volatility are deemed so important that Robert Engle and Clive Granger won the 2003 Nobel prize for their contribution to their development.}
Work on \gls{mgarch} models is still primarily motivated by financial applications.
In such contexts, the prediction task at hand is typically to forecast the next value(s).
This may not typically be the case for neuroscientists (see also \cref{subsec:imputation-benchmark}).
%
\textcite{Lindquist2014} introduced this method for estimating \gls{tvfc}, and it has been used in many studies since~\parencite[see e.g.][]{Choe2017, Preti2017, Foti2019, Hakimdavoodi2020}.
Factored models using low-rank decompositions to deal with high dimensions have been applied to \gls{eeg} sensor data~\parencite{Nakajima2017}.
We discuss similarly factored models for the \gls{wp} in \cref{subsec:model-extensions}.
%
In their early model comparison study, \textcite{Lindquist2014} showed that this method outperforms a naive \gls{sw} approach in many ways (even after post-hoc selection of the optimal fixed window length).
It should not come as a surprise then that methods based on \gls{sw} are not used in finance.
%
A commonly cited reason for the use of \gls{mgarch} in modelling financial time series is that they contain a lot of noise, requiring the use of stochastic methods.
Neuroimaging time series share these data characteristics.

\info[inline]{Paragraph: Describe MGARCH algorithm.}
Many versions and implementations of \gls{mgarch} models exist, as they describe a general \emph{family} of models~\parencite[see][for an extensive overview]{Silvennoinen2009}.
%
The general \gls{mgarch} framework considers a zero-mean vector stochastic process $\mathbf{y}_n \in \mathbb{R}^D$ with time-varying covariance structure $\mathbf{\Sigma}_n$:
\begin{equation}
  \mathbf{y}_n = \mathbf{\Sigma}_n^{\frac12} \mathbf{\eta}_n,
\end{equation}
where $\mathbf{\eta}_n$ is an i.i.d. white noise vector process.
As this is an autoregressive process, $\mathbf{\Sigma}_n$ is only conditioned on all observation up until time $n - 1$.
It assumes that the variances of the individual node time series follow a vector autoregression (VAR) process.
%
Our job as the modeler is then to model $\mathbf{\Sigma}_n$, for which several options exist.
In this thesis, we only consider and report the \gls{dcc} model.
With its introduction, \textcite{Engle2002} showed it to outperform other \gls{mgarch} variants.
It is by far the most common variant used in \gls{rs-fmri} analyses, which allows for better comparison.
Furthermore, \textcite{Heaukulani2019} found the \gls{dcc} variant to outperform the generalized orthogonal~(GO) \gls{garch} model, which is another popular \gls{mgarch} variant.
This latter variant was implemented at an early stage of this work but was consistently outperformed by \gls{dcc} and less robust in its implementation.

\info[inline]{Paragraph: Describe DCC implementation.}
More specifically, throughout this thesis, we implement DCC(1,1)-GARCH(1,1) using the open-source \texttt{R} (version 4.1.0) package \texttt{rmgarch}~\parencite{Galanos2022}.
One of the benefits of this method is that it is well understood and that convenient off-the-shelf implementations exist.
\Gls{mgarch} has several free parameters, although these are often hard to interpret and scale with data dimension to the fourth power~\parencite{Silvennoinen2009}.
\Gls{mgarch} models are known to scale poorly to higher dimensions, although workarounds have been proposed~\parencite[see e.g.][]{Nakajima2017}.
Therefore, \textcite{Gourieroux2009} claimed that these models are typically limited to studying $D < 6$ components.

%%
\subsection{State-based models}\label{subsec:state-based-models}
%%

For completeness, we briefly review state-based models.
These models are sometimes referred to as \emph{switching} models.
They assume a state-space structure of brain activity, consisting of recurring \gls{fc} patterns (see \cref{subsec:brain-states} for more details on this brain state construct).
Brain region dependencies only change when switching brain states.
If such an assumption is made, it makes sense to incorporate this in a model.

The dominant model used for this is the \gls{hmm}~\parencite[see e.g.][]{Vidaurre2017, Ahrends2022}.
It assumes the brain state stochastic process (often called a \emph{chain} in this context) to model is Markovian, meaning the probability of finding oneself in each state in the sequence of all states only depends on the previous state.
However, these states are modeled as `hidden' (i.e.~latent).
The observable process $\mathbf{y}_n$, the \gls{bold} node time series, is then used to infer the underlying hidden brain state process.
