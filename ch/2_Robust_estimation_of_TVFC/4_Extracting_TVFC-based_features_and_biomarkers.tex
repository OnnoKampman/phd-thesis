\clearpage
\section{Extracting TVFC-based features and biomarkers}\label{sec:tvfc-feature-extraction}
%%%%%

\info[inline]{Paragraph: Introduce importance of feature extraction.}
For a given scan, \gls{tvfc} is estimated as a rather large $N \times D \times D$ tensor.
This contrasts \gls{sfc} analyses, where the $N \times D$ data is typically \emph{reduced} in dimensionality to~$D \times D$.
%
In most practical settings and applications, however, we wish to study more interpretable features.
Therefore, it is important to extract features (or \emph{biomarkers} in neuroscientific and clinical contexts) from our predicted correlation structure.
Here we discuss several of such features that will be used in the ensuing chapters.
Details on how exactly they are computed and their relevance for cognition and depression will be discussed in the respective chapters.

\info[inline]{Paragraph: Discuss what we want from extracted features.}
Any data processing step cannot add any information, but merely destroy it.
Therefore, feature extraction should encompass information preservation while removing redundant information (for the task at hand) and noise.
If we desire to use our estimated \gls{tvfc} in some practical application to make predictions, perhaps this estimation can be skipped.
Models can then be trained directly on node time series data.
However, in our context we are interested in the covariance structure itself; as a wiring diagram of the functional interactions within the brain.
%
Extracting features is a crucial step when planning to run any supervised learning algorithm.
In fact, it has been argued that the strong performance of artificial neural networks may be due to their ability to automatically extract meaningful features from data (in contrast to features hand-crafted by humans).

%%
\subsection{TVFC summary measures}\label{subsec:tvfc-summary-measures}
%%

One common way to extract features in \gls{tvfc} neuroimaging is to take edgewise summary measures (or \emph{statistics}) across time.
%
Typical summary measures include the mean~\parencite[analogous to \gls{sfc} estimates; sometimes considered the connection `strength', see e.g.][]{Choe2017} and standard deviation or variance~\parencite[see e.g.][]{Chang2010, Hutchison2013b, Kucyi2013, Kucyi2014, Kaiser2015, Demirtas2016, Choe2017}.
Variance (or standard deviation) is sometimes interpreted to represent connectivity `stability' or `flexibility'~\parencite[see e.g.][]{Allen2014}.
However, we need to be vigilant with vague terminology and premature interpretations.

More precisely, \gls{tvfc} means are calculated as
\begin{equation}
  \mu_{i,j} = \frac{1}{N} \sum_{n=1}^N \mathbf{\Sigma}_{n,i,j},
\end{equation}
and \gls{tvfc} variances are calculated as
\begin{equation}
  \sigma^2_{i,j} = \frac{1}{N} \sum_{n=1}^N (\mathbf{\Sigma}_{n,i,j} - \mu_{i,j})^2,
\end{equation}
for the edge between nodes $1 \leq i \leq D$ and $1 \leq j \leq D$.

Here we propose an additional summary measure, the \gls{tvfc} \textbf{rate-of-change}, defined as the mean absolute relative difference per subsequent time steps:
\begin{equation}
  r_{i,j} = \frac{1}{N-1} \sum_{n=2}^N | \frac{\mathbf{\Sigma}_{n,i,j}}{\mathbf{\Sigma}_{n-1,i,j}} - 1 |,
\end{equation}
for the edge between nodes $1 \leq i \leq D$ and $1 \leq j \leq D$.
This summary measure captures how smooth a time series is over time (i.e.~the smoothness of the estimated \gls{fc} time series in our case).
It is more informative of \gls{fc} frequency amplitudes and is akin to \gls{fc} `variability' as described in \textcite{Allen2014}.
To illustrate its relevance, this summary measure can distinguish two sine waves with identical mean and variance, yet oscillating at different frequencies (see e.g. \cref{fig:synthetic-covariance-structures}).
As such it is complementary to the other two summary measures.

%%
\subsection{Brain states and related metrics}\label{subsec:brain-states}
%%

Another common way to reduce the dimensionality of \gls{tvfc} estimates is to assume that the estimated covariance structure at each time point can be characterized as a certain `brain state' (or `\gls{fc} state')~\parencite{Kringelbach2020}.
Such states are short-term, recurring spatial activity \gls{fc} patterns across time and subjects.
In our context, they are characterized by an \gls{fc} correlation matrix.
%
The existence of such a state-space structure in the brain has been shown to be a valid view~\parencite{Deco2015}.
%
These states can be insightful on their own or can be used as extracted features.
For example, \textcite{Rashid2016} showed that schizophrenia patients spend more time in low-contrast states.
One exciting aspect of the construct of brain states is that they can be synthesized across species and imaging modalities (e.g.~with microstates in \gls{eeg}, see \textcite{Allen2014} for further discussion).
As such they can serve as a common language to bridge multiple levels of neuroscientific research.

Brain states are typically extracted either by using $k$-means clustering on estimated \gls{tvfc}~\parencite[see e.g.][]{Allen2014, Abrol2016, Zhi2018, Hakimdavoodi2020} or using models (usually \glspl{hmm}, see \cref{subsec:state-based-models}) that have this states assumption baked into their definition~\parencite{Lurie2020}.
In such latter cases we do not estimate the full \gls{tvfc} tensor.
Thus, certain information is lost in the process.

Here we extract brain states from estimated \gls{tvfc} using the original $k$-means algorithm described in \textcite{Lloyd1982}.
All estimated correlation matrices for all~$N$ time steps and all subjects are concatenated and fed into this algorithm.
The $k$-means algorithm requires a pre-specified number of clusters~$k$.
One common trick to determine this number is to plot an `elbow curve'~\parencite[see e.g.][chapter 5.5]{Everitt2011}.
Such a plot computes the summed distance between each correlation matrix and its closest (i.e.~assigned) basis state for a range of values of~$k$.
The optimal value of~$k$ is then chosen as the one where this curve has an `elbow'; that is, after which the curve is relatively linear and after which adding another cluster does not result in a big decrease thereof.
%
Perhaps surprisingly, most studies using brain states to summarize the brain's activity find a relatively small number of distinct states, for example three in \textcite{Choe2017, Dini2021} and 12 in \textcite{Vidaurre2017}.

Even though the \gls{wp} is not constrained in estimating \gls{tvfc} in grid-like fashion, we do this for extracting brain states to allow for better comparison to other methods.
However, we could estimate as many covariance matrices at as high of a temporal resolution as we like for this task.
%
Furthermore, some studies have relaxed the assumption that participants are assigned to a single state and learn a weight vector over all basis states instead for each time step~\parencite{Leonardi2014}.
Such an approach again blurs the boundary between the continuous modeling of brain activity and viewing the brain as a state-space system.
The latter view is still debated~\parencite[see][for an excellent discussion on the potential and shortcomings of the brain states framework]{Keilholz2017}.

%%
\subsubsection{Brain state metrics}
%%

If our estimation method produces brain states, we can also extract interpretable features from these.
These include global metrics such as brain state switching rate (how frequently participants switch between states),
and state-specific metrics such as occupancy rate and dwell time (the relative time participants spend in each state).

%%
\subsection{Graph theoretic analysis}\label{subsec:graph-theoretic-analysis}
%%

Yet another common feature set extracted from \gls{tvfc} is to (mathematically) represent brain network connectivity as a graph and then compute graph theoretic metrics~\parencite{Sporns2011}.
Such approaches can be used to study topological properties of brain networks and the properties of connectivity in the brain.
In graph theory, brain \glspl{roi} are represented as \textbf{nodes}.
The \textbf{edges} between brain regions can take many forms.
As mentioned before, throughout this thesis we will use the terminology of nodes and edges to describe brain regions and their respective connections.

It is common to use \gls{rs-fmri} and define edge strength as a given \gls{fc} metric (often, and in our case as well, correlation).
Edges can be thresholded and binarized, but weighted graphs are also common.
Standard graph approaches are typically used on graphs defined by \gls{sfc}, such as decomposing networks into modules~\parencite{Betzel2016}, detecting hubs, and computing graph metrics.
Such graph metrics fall into one of three categories: global properties (e.g.~global efficiency and path length), modularity, and local properties (e.g.~clustering).

Neuroscientists have increasingly adopted techniques from the field of network science.
Graph theoretic concepts such as centrality, efficiency, network modularity~\parencite{Zalesky2014}, and community structure have been extracted from brain networks.
For dynamic networks, such changes have been related to learning~\parencite{Bassett2011}.
Such dynamic changes have had a significant impact on basic neuroscience in understanding concepts such as the dynamics of integration and segregation in the brain, including in understanding disorders such as depression~\parencite{Gong2015}.

Some controversy remains in graph theoretic studies.
First, some graph metrics may not be applicable to the data at hand.
Graph \emph{efficiency}, for example, assumes causal connections between nodes, and may not be valid in (undirected) graphs defined by \gls{fc}~\parencite{Chen2017}.
Secondly, the interpretation of graph metrics often leaves much to speculation.

%%
\subsubsection{Dynamic graph theoretic metrics}
%%

As this thesis is concerned with dynamics of brain activity, we would want to study time-varying properties of graphs and graph theoretic metrics.
However, this field has not received much attention yet.
%
In the simplest case these could be summary measures of graph metrics across time again.
\textcite{Zalesky2014} demonstrated the synthesis of \gls{tvfc} with graph theory.
Some dynamic graph studies have been run~\parencite{Bassett2011, Bassett2013}.
This is a newer approach, but we may take inspiration from the field of temporal network theory~\parencite{Holme2012, Yu2015, Thompson2017}.

%%
\subsection{Feature extraction from models}\label{subsec:model-features}
%%

Instead of merely describing or estimating covariance, such as the \gls{sw} approach, the \gls{wp} is a generative model.
Like \gls{dcc}, it is a model-based approach.
Thus, we have a full model for each scan and subject.
Assuming we have a good model of the underlying process, we may be able to extract relevant process or subject features.

An example would be the learned \gls{wp} kernel lengthscale $l$, see \cref{eq:matern}, which could be extracted from the trained models and compared between subjects and scan times.
The reverse may also be interesting.
For example, in a similar Bayesian model approach, \textcite{Li2019a} defined the prior for their \gls{gp} lengthscale based on their expectation to find frequency dynamics close to the theta range (4--12 Hz).
%
With this in mind, we can make model choices that explicitly return interpretable and cognition-relevant features.
However, \glspl{gp} (and related models) are not considered to be good at feature extraction, since we encode a lot of the data structure into the kernel choice.
This leaves only several hyperparameters that dictate the data description.
Instead, deep learning models might be better feature extractors~\parencite[and have proven useful in other neuroscientific fields, see e.g.][]{Richards2019}, although these typically require a lot of data.
Overall, we consider model-based approaches a promising direction for feature extraction, but one that requires much work still.
