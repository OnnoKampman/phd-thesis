\clearpage
\section{Summary of presented work}
%%%%%

We started off this thesis describing the open questions and issues when choosing how to estimate \gls{tvfc} from \gls{fmri} time series.
We also motivated the study of depression, a human tragedy affecting millions.
Throughout the thesis we have motivated the importance of robust science, especially in the context of making high-stakes claims about important topics (such as depression).
The specific thesis contributions are summarized once more here.

In \cref{ch:methods} we investigated how to robustly estimate \gls{tvfc}.
We proposed a more principled way of running \gls{sw} and proposed a novel method to the field: the \gls{wp}.
This is a covariance matrix stochastic process that has recently become computationally tractable.
Moreover, we carefully considered how to compare \gls{tvfc} estimation methods.
We translated principles from the field of machine learning to design a comprehensive suite of data science tasks, i.e.~benchmarks.

In \cref{ch:benchmarking} we discussed all these benchmarks and showed how our new method broadly outperformed competitive baselines: both \gls{dcc} and \gls{sw} approaches.
We also compared these to \gls{sfc}, and were able to profile \gls{tvfc} beyond simply picking the optimal method.
Benchmarks included simulations, subject phenotype prediction, test-retest studies, brain state analyses, external task prediction, and a range of qualitative method comparisons.
A new benchmark based on cross-validation was introduced, that can be run on any data set.

In \cref{ch:ukb} we studied how \gls{tvfc} can be used as a lens through which to study depression as a functional disorder.
Several differences were found between depressed and healthy control cohorts.
The study is run on thousands of participants from the UK Biobank, yielding unprecedented statistical power and robustness.
Individual brain regions as well as \glspl{fn} connectivity were investigated.
Depressed participants show decreased global connectivity and increased connectivity instability (as measured by the temporal characteristics of estimated \gls{tvfc}).
By defining multiple depression phenotypes, brain dynamics are found to be affected especially when patients have been professionally diagnosed or indicated to be depressed during their \gls{fmri} scan but were less or not at all affected based on self-reported past instances and genetic predisposition.
It was demonstrated that choosing a different \gls{tvfc} estimation method would have changed our scientific conclusions.
This sensitivity to seemingly arbitrary researcher choices highlights the need for robust method development and the importance of community-approved benchmarking.
