\clearpage
\section{Concluding remarks}
\label{sec:concluding-remarks}
%%%%%

When I started my PhD I did not have much of a background in neuroscience or psychology.
In this early stage I attended an excellent workshop on psychology as a robust science, taught by Amy Orben.
It had a great impact on me.
On some level I consider myself a child of the reproducibility crisis in psychology.

Consider the following two stories.

In 1998 a fraudulent scientist published an uncontrolled study run on 12 children in the Lancet that claimed a link between MMR vaccines and autism.
Vaccine skepticism has grown considerably since then with deadly consequences.
Even though a study including more than a million participants showed no effect, and even as his original co-authors retracted the paper, the damage was done.

In 2012 a TED talk about power poses was uploaded to the internet and it quickly became the website's most-watched talk ever.
However, it turned out that the research it was based on did not replicate.
The TED website now even has a disclaimer saying this research is debated, in the style of a social media platform adding disclaimers to stories from dodgy sources.

What is the moral of these stories?

Commitment to being a scientist does not just mean that we should try to speak the truth as best as we can.
As elegantly put by Richard Feynman: it is about bending over backwards to provide as much information as possible as to why we may be wrong.

This thesis certainly does not provide any conclusive evidence.
However, it is my hope that it takes us a small yet \emph{robust} step into the right direction.
Depression is a topic that desperately needs better understanding.
I hope the overal style of this work contributes to a science that is trustworthy and cumulative in nature.
