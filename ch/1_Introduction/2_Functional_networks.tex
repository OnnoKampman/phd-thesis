\clearpage
\section{Functional networks}\label{sec:functional-brain-networks}
%%%%%

\info[inline]{Paragraph: Introduce the concept of functional networks.}
Brain regions have empirically been found to cluster and interact to adapt to a certain task at hand, forming \emph{networks} (also referred to as `circuits' or `systems')~\parencite{Fox2007}.
Cognitive tasks are not just performed by isolated brain regions, but rather by such networks, i.e.~linked collections of brain regions~\parencite{Bressler2010}.
%
These networks have been identified from \gls{fmri} \gls{bold} signals from scanning subjects presented with external tasks.
Moreover, it has been shown that such large-scale networks exist at rest, and that these strongly resemble those found in task paradigms~\parencite{Smith2009}.\footnote{In neuroimaging, `rest' usually refers to the absence of an external stimulus or task, leaving mental activity relatively unconstrained. Signals measured at rest are sometimes referred to as `intrinsic' or `spontaneous', but rather confusingly in neuroimaging this sometimes refers to an actual signal of interest and sometimes to just noise.}
In such cases, these networks are referred to as \glspl{rsn} or \glspl{icn}.
However, this name can cause confusion, because networks with similar extents can be found during task executions.
Therefore, we opt to simply use the term `functional network'~\parencite[FN; see also][]{Finn2021}.
Viewing the brain as a superposition of networks is yet another level up in abstraction, beyond looking at individual brain regions.

\info[inline]{Paragraph: Describe several common functional networks that have been discovered.}
Depending on the analysis method, several such \glspl{fn} have been identified in humans, typically ranging from 2 to 20~\parencite{Yeo2011, Heine2012, Glomb2017}.
One particularly impactful \gls{rsn} study found 20 networks~\parencite{Smith2009, Laird2011}, ten of which showed strong overlap between resting-state and task-based data networks.
We will return to these ten networks in \cref{ch:benchmarking}.
They are printed in \cref{fig:brainmap-functional-networks}.
%
There are several core networks that are stable, widely recognized, and typically found across network extraction methods and data sets~\parencite{Uddin2019}.
These \glspl{fn} include the \gls{dmn}~\parencite{Raichle2001, Raichle2007, Vatansever2015}, the \gls{cen}~\parencite{Rogers2004},\footnote{The \gls{cen} is also known as the \gls{ecn}, cognitive control network (CCN), or frontoparietal network.} and the \gls{sn}~\parencite{Drevets2000, Seeley2007}.\footnote{In neuroscience, the term `salience' refers to the property of being noticed and/or paid attention to. Any biological agent needs to carefully assess what to spend their limited perceptual and cognitive resources on (both internal and external).}
These networks are respectively broadly associated with rest, cognition, and emotion~\parencite{Uddin2019}.
%
There is some disagreement about the validity of these networks and what their constituents are.
Furthermore, some brain regions can be part of multiple \glspl{fn}, and some brain regions are not considered to be part of any \gls{fn}.

\info[inline]{Paragraph: Discuss scientific insights gained from functional network studies.}
Viewing cognitive function through the lens of \glspl{fn} has proven fruitful and valid.
For example, \textcite{Vidaurre2017} showed that such large-scale brain networks are hierarchically organized and heritable.
Furthermore, they show that the switching between such networks is not random, and the time a subject spends in each state is predictive of cognitive traits.
