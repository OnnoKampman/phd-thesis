\clearpage
\section{The trouble with estimating TVFC}
%%%%%

\info[inline]{Paragraph: Overview, scope, and importance of problem of TVFC estimation.}
One of the main reasons for the lack of understanding of \gls{tvfc} as a construct and its cognitive relevance is that the field still lacks a robust way of estimating \gls{tvfc}~\parencite{Foti2019, Lurie2020}.
Many estimation methods are used in practice, and many more have been proposed.
\gls{tvfc} estimations vary wildly across different estimation methods, and, as we will see, this results in different predictive power of subject measures (including clinical measures).
Consequently, experimental and scientific conclusions are heavily influenced by the (seemingly arbitrary) choice of estimation method.
This compounds onto the already large number of reseacher `degrees of freedom' present in \gls{fmri} analyses~\parencite{Gelman2013, Dafflon2022}.
This motivates the careful and robust development of \gls{tvfc} estimation methods.

\info[inline]{Paragraph: Introduce common estimation methods.}
Common \gls{tvfc} estimation approaches include \gls{sw} methods~\parencite{Sakoglu2010, Chang2010}, multiplication of temporal derivatives~\parencite{Shine2015}, phase coherence models~\parencite{Glerean2012}, \gls{mgarch} methods~\parencite{Lindquist2014, Choe2017, Xie2019}, a range of customized Bayesian models~\parencite[see e.g.][]{Taghia2017, Lan2017, Warnick2018, Li2019b, Ebrahimi2020}, wavelet methods~\parencite{Park2014, Zhang2016}, and phase synchronization models~\parencite{Varela2001, Glerean2012, Demirtas2016, Honari2021}.
Each of these has a range of variants and tweaks as well.
We need to be careful with comparing these approaches head-to-head, as many of these methods extract a slightly different aspect of \gls{tvfc}, where some focus more on frequency-space information, others model change points better, and some are autoregressive whereas others are not.
Moreover, it is often not clear what connectivity measure is of interest for answering a particular scientific question.
This lack of clarity is another major hurdle in the field.
%
Part of the problem is that the target audience of proposed methods differs.
Many of the more advanced model proposals address a very technical audience.
This makes practical adoption unlikely.
Additionally, such more principled and thoughtful methods are often published without clean and stable code.
These factors have led to a misalignment between thoughtful modeling experts that develop better methods and their envisioned end users, who (understandably) still often default to the simple and practical \gls{sw} approach.
%
In \cref{sec:established-methods} we will go into more technical detail on the methods considered in this thesis.
%
Although we only consider time domain methods, time-frequency parameters can still be extracted from learned model parameters, as we shall see.
Even though not every method is considered, when setting up the benchmarks it should be relatively straightforward to include another method to our comparison framework.

\info[inline]{Paragraph: Explain why method selection is hard.}
The lack of a ground truth correlation makes method selection a hard problem.
This explains why the field has not settled on a single approach.
%
In their review, \textcite{Lurie2020} noted the pitfall of studying \gls{rs-fmri} \gls{tvfc} of lacking clear benchmarks.
They also notd that \gls{rs-fmri} has already gone through similar controversies in its early days.
This invites us to learn from its respective journey as a field.

\info[inline]{Paragraph: State how we address this problem.}
How do we then determine how to estimate \gls{tvfc}?
This thesis aims to answer this question.
In short, we propose a suite of data science problems; a range of \emph{benchmarks} (see \cref{sec:benchmark-framework}).
These benchmarks are designed to capture and illuminate various estimation method attributes, strengths, and weaknesses.
Overall, the results then provide an overview of method performance and failure modes.
