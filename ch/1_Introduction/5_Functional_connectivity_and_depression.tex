\clearpage
\section{Functional connectivity and depression}\label{sec:fc-depression}
%%%%%

\info[inline]{Paragraph: Introduce the general study of depression.}
In this thesis we study depression as a functional and connectivity disorder, through the lens of \gls{tvfc}.
But what do we mean by depression?
How does depression affect the brain?
And more specifically, how does depression affect \gls{fc} in the brain?
Can \gls{fc} be used to assign credit or discredit to various theories of depression?
In this thesis we argue that depression is a particularly good disease to study through the lens of \gls{tvfc}.
\gls{fc} has the potential of offering new diagnostic value in neuropsychiatric disorders, where typical \gls{fmri} activations are often small~\parencite{Fornito2012}.
The rest of this section reviews the current understanding of what depression is, why it is important to study it, what subtypes exist, what symptoms typically occur, how it affects the brain, and how it affects \gls{fc} in the brain.
Of course, this will be a limited overview of all research and perspectives on depression, but will include the most relevant background information for the study in this work.

%%
\subsection{What is depression?}\label{subsec:depression}
%%

\info[inline]{Paragraph: Overview of depression burden and motivation to study it.}
Depression is a human tragedy: it is absolutely crippling, it is pervasive, and it is global.
The most recent \gls{who} estimate (for 2021) puts the number of people worldwide living with a proverbial `black dog' at 280 million.
The burden of depression (and other neuropsychiatric disorders) on societies and their healthcare systems barely needs elaboration.
Even more worrisome is that its disease burden and prevalence are growing.
Stigma, heterogeneity of symptoms, and lack of understanding of causes and brain and social mechanisms have meant that treatment of this disorder (or umbrella of disorders) remains insufficient.
Depression is slightly different for everyone and remains difficult to conceptualize.
We may call it a disease, illness, or disorder, but we may also view it as an \emph{experience} instead.

\info[inline]{Paragraph: Introduce depressive disorders and describe MDD.}
This thesis is mainly concerned with \gls{mdd}, the most common of all depressive disorders.
It is important to distinguish between three types of depression: the everyday, colloquial use of the word depression; a longer period of sadness after a traumatic life event (a \emph{reactive} depression); and \textbf{major depression}, which is characterized by \emph{persistent} sadness over prolonged periods of time~\parencite{Otte2016}.
Going forward we refer to the latter when we talk about depression.
Two common ways of diagnosing (i.e.~categorizing or classifying) depression are based on standard diagnostic (category-based) frameworks: the \gls{dsm} and the \gls{icd}.\footnote{The \gls{icd} defines mental disorders as ``clinically recognizable set of symptoms or behaviors associated in most cases with distress and with interference with personal functions''~\parencite{WHO1992}.}
The \gls{dsm} criteria for major depression are shown in Box~\ref{box:depression}.
We shall return to these criteria in \cref{subsec:cohort-stratification} for defining participant cohorts.
The \gls{icd} criteria are similar, requiring three criteria to be met: persistent low mood or loss of interests; happening for most of the time on most days for at least two weeks; and experiencing four or more other symptoms like disturbed sleep, concentration, appetite, guilt, self-blame, low self-esteem, agitation, and/or suicidal thoughts.
More broadly, depression endophenotypes\footnote{Endophenotypes, or `intermediate phenotypes', refer to heritable traits used to more robustly define behavioral symptoms into phenotypes. Similar terms are `biological marker' or \emph{biomarker} and `subclinical trait', although these are typically not used to refer to genetic components.} and cardinal symptoms include anhedonia, anergia, anxiety, rumination, changes in appetite and sleep patterns, strong and persistent feelings of guilt and grief, and, most tragically, self-injury~\parencite{Goldstein2014, Pizzagalli2014}.
Although core symptoms are typically present, depression is not a consistent syndrome with a fixed set of symptoms.
In fact, \textcite{Fried2015} found over 1,000 unique symptoms in a cohort of about 3,700 patients~\parencite[see also][]{Fried2015b}.
\Gls{mdd} not only affects mood and affective processing but is also involved with a range of cognitive dysfunctions.

\begin{mybox}[floatplacement=t,fontupper=\footnotesize,fontlower=\footnotesize,label={box:depression},colback=White]{Depression and its symptoms}

  The latest version of the \gls{dsm} (version 5, released by the American Psychiatric Association in 2013) outlines the following symptoms and criteria for diagnosing \gls{mdd}.
  The individual at hand must experience five or more of the following symptoms, persisting for at least two weeks.
  Symptoms must be present nearly every day, for most of the day.
  One of the two `core' symptoms, depressed mood (1) or (2) loss of interest or pleasure, must be present.

  \tcblower

  \begin{enumerate}
    \item Depressed, sad, and/or low mood.
    \item Diminished interest or pleasure in all, or almost all, activities.
    \item Significant weight loss or weight gain, or changes in appetite.
    \item Sleeping too much, too little, or not well (insomnia).
    \item A slowing down of thought and a reduction of physical movement (observable by others, not merely subjective feelings of restlessness or being slowed down), that is psychomotor agitation or retardation.
    \item Fatigue or loss of energy.
    \item Feelings of worthlessness or excessive or inappropriate guilt.
    \item Diminished ability to think or concentrate, or indecisiveness.
    \item Recurrent thoughts of death, suicidal ideation (with or without a specific plan), or a suicide attempt.
  \end{enumerate}

\end{mybox}

\info[inline]{Paragraph: Discuss causes and enviromental factors for depressive disorders.}
What causes depression?
As there are various subtypes of depression, this varies.
However, commonly depressive episodes are predated by traumatic, adverse, and negative life events~\parencite{Kessler1997, Monroe2008}.
When such events happen at a developmental age, they can disproportionally impact neurobiological systems, and lead to a higher probability of developing depression later in life.
Perhaps the right question is not what causes depressive episodes, but what makes some individuals more \emph{resilient} in the face of stressors to be able to cope and recover.
As such it is common to talk about `risk' or `contributing' factors (such as genetics, early life experiences, socioeconomic status, and environment), instead of `causes'.
Most prevention efforts would focus on managing exactly these contributing factors.

\info[inline]{Paragraph: Discuss genetic contribution to MDD.}
There is some evidence for a genetic basis or predisposition to depression.
This will be relevant in \cref{subsec:cohort-stratification}.
We know that \gls{mdd} has a heritable component.
Genetic risk for \gls{mdd} is polygenic, meaning a variety of genes are involved, and the exact mechanisms are yet to be uncovered~\parencite{Hyman2014}.
This is likely due to the heterogeneity of depressive symptoms as well.
Moreover, much of depression risk may be due to other genetic factors.
Higher overall cognitive function, for example, could lead to higher socioeconomic status, which in turn could lead to a healthier diet and an increased sense of safety and control in the world (which in turn have been linked to lower depression risk).
Genetic risk has often been described as influencing cognitive biases and thus \emph{resilience} to stressors.
The most important take-away from genetic studies is that genes are about vulnerability and resilience to depression and not about inevitability.
%
In a large cohort study, \textcite{Garcia-Gonzalez2017} failed to find a robust genetic contribution to treatment response.

\info[inline]{Paragraph: Describe cognitive effects of MDD.}
Before looking at the brain and the impacts of \gls{mdd} on \gls{fc}, we give an overview of changes in cognition and behavior.
These will be referred to in \cref{sec:ukb-discussion}.
Deficits in memory systems, attention, learning, processing speed, and decision-making are common among \gls{mdd} patients.
\textcite{Rock2014} found especially executive function\footnote{In neuroscience, \textbf{executive function} generally refers to functions related to planning, focus, sticking with instructions, and multi-tasking~\parencite{Banich2009}.}, memory, and attention affected by \gls{mdd}.
Dysfunction is linked to a range of cognitive and affective biases.
A core affective bias is toward paying attention to the negative, or only remembering the negative~\parencite{Pulcu2017}.
For example, depressed individuals forget negative information at a slower rate~\parencite{Power2000, Joormann2010}.

\info[inline]{Paragraph: Discuss integrated models of depression.}
Key to all of this is to find ways to \emph{integrate} or \emph{unify} the various perspectives on depression.
Several proposals have been made to build integrated models of depression.
Most of these agree that we need a bridge between the psychological perspective (the one that `makes sense' but we cannot do modern science on) and the biological perspective (the one that we can measure and work with but is often too far removed from the human experience).
\textcite{Akiskal1973} discussed ways to integrate such psychological and biological views of depression.
Their proposed framework integrates several depression characterizations; metapsychological (Freud's ``aggression-turned-inwards'' and the ``object-loss'' models), the ``reinforcement'' model, and the biological (``biogenic amine'') model into a common pathway of ``functional derangement of the mechanisms of reinforcement''.
\textcite{Pizzagalli2014} proposed that anhedonia is the key feature of depression, and proposes an account of anhedonia, \gls{da} (reward systems), and the (internal) massive stress responses and heightened stress hormone levels found in depressed patients.
More recently, \textcite{Beck2016} proposed that depression can be viewed as ``an adaptation to conserve energy after the \emph{perceived loss of an investment in a vital resource} such as a relationship, group identity, or personal asset.''
They highlight that these are mediated by brain regions involved in cognition and emotion regulation: the \gls{amg}, \gls{hpc}, and \gls{pfc}.
According to this proposal, depression can be viewed as an ``evolutionary program'' for conserving energy, that just so happens to have become maladaptive in contemporary life.\footnote{Other maladies can also be attributed to such unfortunate evolutionary left-overs. Instinctive hoarding of sugar and information had evolutionary advantages, but wreaks havoc in modern life.}
Overall, many of these existing grand theories share a lot of common ground.
Most descriptions gear toward perturbations in reinforcement processing, negative affective bias~\parencite{Pulcu2017}, negative feedback loops, stress, and associated neurochemical pathways.
However, at the time of writing most of these are still quite general and fail to make concrete, falsifiable predictions, crucial for the development of strong theory.
It also remains to be seen whether a single model will be able to describe all clinical cases.

\info[inline]{Paragraph: Discuss treatment options for MDD.}
That brings us to the treatment of depression.
Importantly, knowing what works to treat depression and what does not can also shed light on what the condition entails.
Treatment options for \gls{mdd} generally are pharmacological intervention and/or one of the many types of (psycho)therapy~\parencite{Otte2016}.
Antidepressant medication is usually meant to increase the concentration of a certain neurotransmitter in the brain, most commonly serotonin.
In the case of serotonin, these antidepressants are called \gls{ssri}.
Interest in therapies using psilocybin~\parencite{Carhart-Harris2016, Luppi2021, Daws2022, Singleton2022} and ketamine~\parencite{Krystal2019, Kotoula2021} has spiked in recent years as well, but the jury is still out on the efficacy of such treatments.
Depression is seen as decreasing brain state entropy, which psychedelics can alleviate, as it lies on the opposite side of a spatiotemporal dynamics spectrum~\parencite{Vohryzek2022}.
%
There is an intense debate in society at large on how best to treat depression, some pointing predominantly to medication, and others to social approaches.
Part of this controversy stems from the fact that depression is still poorly understood, and it is likely grouping together many types of depression.
For example, medication seems to work well for some but has no effect on others.
The latter are sometimes called `treatment-resistant', but they may well suffer from a different subtype of depression, where neurobiologically distinct domains are collapsed into a simple diagnostic index.
Overall, there are many things that can help those with depression.
However, one of the crucial issues is that not all these things help everyone, and matching the right support to the right person is hard.
Each patient is characterized by a unique mixture of medical history, personality, comorbidities, socioeconomic environment, and many other factors~\parencite{Trivedi2006}.
Here lies the challenge of the treatment of depression in society: how do we provide care with the required level of personalization, yet to millions of people at the same time?

%%
\subsection{Depression and neuroimaging}\label{subsec:fc-neuroimaging}
%%

Neuroimaging has the potential to offer unique insight into the mechanisms of depression.
A lot of neuroimaging can be considered a mapping exercise.
Starting from how the brain is anatomically characterized, to mapping literal wiring diagrams such as white matter tracts, but also including the \gls{fc} mapping exercise as we study here.
Once we have a good map, it is natural to ask whether we can find individual landmarks that are unique to a disease.

Structural scans have shown that the neuroanatomy in depressed patients is affected~\parencite{Drevets2000}.
Multiple brain region volumes are either increased or decreased~\parencite{Sacher2012, Schmaal2020}.
Grey matter volumes are \emph{reduced} in the \gls{amg}, \gls{pccx}, \gls{dmpfc}, and \gls{hpc}.
However, there are conflicting findings, and \gls{amg} volume may be increased or decreased based on individual specifics.
Volumetric increases have been reported for the insula, middle frontal gyrus, superior frontal gyrus, and the thalamus.\footnote{In neuroanatomy, a \textbf{gyrus} refers to the ridges of the cortex surface, opposed to a \textbf{sulcus}, which refers to the respective furrow of the folded cortex.}

However, recent meta-analyses have come to dispute the reliability and clinical relevance of many such findings.
A recent large (1809 participants) study found very modest predictive power of (univariate) neuroimaging modalities (\gls{mri}, \gls{dti}, \gls{rs-fmri}, and \gls{tb-fmri}) of \gls{mdd}~\parencite{Winter2022}.
They found environmental factors such as social support and childhood maltreatment to have much more predictive power.
Similar sentiments were echoed in \textcite{Nour2022}.
At present, neuroimaging plays little to no role in clinical decision-making~\parencite{Kapur2012}.

%%
\subsection{Functional connectivity in psychiatric disorders}\label{subsec:fc-depression}
%%

\info[inline]{Paragraph: How can we relate functional connectivity to disorders? What can this teach us about these disorders? What value do these analyses have in a clinical setting?}
Whereas some brain disorders and mental health conditions can be traced back to a dysfunction in a particular brain region (e.g.~inflammation, neurodegeneration,\footnote{Neurodegenerative disorders refer to progressive loss of neural structure and function. Common disorders in this category include \gls{ad} and \gls{pd}.} or physical trauma), others are better understood as dysfunction in brain region \emph{function} and/or \emph{interaction} between otherwise seemingly healthy individual brain regions.
Mood disorders\footnote{In psychiatry, a \textbf{mood} or \textbf{affective disorder} refers to any depressive or bipolar disorder.} especially have been suggested to be functional (related to dynamic connectivity patterns) rather than structural disorders~\parencite{Piguet2021}.
\Gls{fc} is a particularly useful framework to study such aberrations in connectivity with, witnessed by the vast number of studies studying depression through this lens.

\info[inline]{Paragraph: Discuss sFC in neurological and psychiatric disorders.}
Most of such psychiatric studies employ \gls{sfc}.
These estimates have demonstrated potential and promise to be used in clinical settings~\parencite{Cole2014, Deco2014, Parkes2020}, or at least have some predictive power.
For example, it has been shown to be useful as biomarkers for depression~\parencite{Drysdale2017}.
The promise of using connectomics for understanding neuropsychiatric disorders has been expressed many times~\parencite{Deco2014}, and many \gls{sfc} studies have been conducted.
Building on top of advances in network neuroscience has shown more potential for clinical discovery from \gls{fc}~\parencite{Lydon-Staley2018}, where network features can be used as biomarkers of disease~\parencite{Bassett2009}.
Another data point that highlights the promise of studying \gls{mdd} through the perspective of \gls{fc} is that such connectivity and networks have been found to re-normalize after anti-depressants treatment.

\info[inline]{Paragraph: Discuss FNs in neurological and psychiatric disorders.}
\Glspl{fn} are often affected by neuropsychiatric disorders, even if their individual brain region constituents appear normal.
Such disorders are therefore increasingly studied as \emph{network} disorders~\parencite[see][for a review on depression]{Mulders2015}.
Instead of a single brain region not functioning properly, there is an aberration in the integration and segregation of brain regions.
These changes in \gls{fn} are believed to contribute to or be caused by cognitive changes from mental illness.
Many mental illness conditions have been postulated to occur with large-scale disruptions, driven by neurotransmitter dysfunction, of whole-brain systems.
Even though whole-brain \glspl{fn} are found to be highly similar across groups with or without a range of mental illnesses, the subtle differences that \emph{do} occur are meaningful in the sense that they are predictive of diagnosis~\parencite{Spronk2020}.
This makes intuitive sense as well: a piano does not need major disruption to ruin a classical piece, one key being out of tune is sufficient~\parencite[see also][for a discussion of small effect sizes]{Paulus2019}.
Perhaps this is even encouraging.
If small network alterations can result in mental disease, a small intervention can bring someone's functional architecture back on track.

\textcite{Mulders2015} found the main networks to be involved and affected in depression to be the \gls{dmn}~\parencite{Berman2011, Demirtas2016, Wise2017, Yan2019, Zhao2019, Zhou2020}, \gls{cen}~\parencite{Zhao2019}, and \gls{sn}~\parencite{Manoliu2014}.
These are also generally the networks most studied and will be the ones considered in this thesis.
The exact makeup of these networks varies across studies.
A rough overview of each network is provided here (the more precise implementational details will be provided in \cref{subsec:ukb-fn-analysis}).

The \gls{dmn} primarily consists of the \gls{mpfc} and \gls{pcc}, as well as the (para)hippocampal areas, precuneus (cortex), and angular gyrus~\parencite{Andrews-Hanna2010}.
It is often described as the neurological basis for `the self' and is attributed to functions like self-referential thinking~\parencite{Sheline2009}, cognitive flexibility~\parencite{Vatansever2016}, mind-wandering, memory processing and rumination, theory of mind, emotion regulation, and as storage of autobiographical information.
It is connected to the \gls{amg} and \gls{hpc}~\parencite{Andrews-Hanna2014}.

The \gls{cen} primarily consists of the lateral \gls{pfc}, posterior parietal cortex (PPC), \gls{dlpfc} (especially middle frontal gyrus), \gls{dmpfc}, and posterior parietal regions~\parencite{Rogers2004}.
It is associated with cognitive processes and functions, like working memory and attention.

The \gls{sn} primarily consists of the \gls{ai} and (dorsal) \gls{acc}, with some adding the \gls{amg}, frontoinsular cortex, temporal poles, and striatum~\parencite{Seeley2007, Menon2010, Beck2016}.
The \gls{sn} is a key network in cognitive flexibility~\parencite{Dajani2015}.

\info[inline]{Paragraph: Discuss TVFC in neurological and psychiatric disorders.}
What about the dynamics of \gls{fc}?
The promise of \gls{tvfc} has been highlighted more recently as well in neurodegenerative conditions~\parencite{Filippi2019}.
More relevant information is contained in \gls{tvfc} compared to \gls{sfc}.
\gls{tvfc} may be especially relevant for dynamic brain disorders like schizophrenia~\parencite{Jin2017}.

\info[inline]{Paragraph: Discuss graph topology in neurological and psychiatric disorders.}
Graph topology and network neuroscience have also been suggested to shed more light on neurological and psychiatric conditions~\parencite{Fornito2013}.
Connectomic graph theoretic approaches to depression have found smaller path lengths and higher global efficiency~\parencite{Zhang2011}.
This has been interpreted as a shift toward brain network randomization~\parencite{Gong2015}.
